% T I T L E   P A G E
% -------------------
% Last updated Nov 1, 2016, by Stephen Carr, IST-Client Services
% The title page is counted as page `i' but we need to suppress the
% page number.  We also don't want any headers or footers.
\pagestyle{empty}
\pagenumbering{roman}

% The contents of the title page are specified in the "titlepage"
% environment.
\begin{titlepage}
        \begin{center}
        \vspace*{1.0cm}

        \Huge
        {\bf Inferential Role Semantics for Natural Language}

        \vspace*{1.0cm}

        \normalsize
        by \\

        \vspace*{1.0cm}

        \Large
        Peter Blouw \\

        \vspace*{3.0cm}

        \normalsize
        A thesis \\
        presented to the University of Waterloo \\ 
        in fulfillment of the \\
        thesis requirement for the degree of \\
        Doctor of Philosophy \\
        in \\
        Philosophy \\

        \vspace*{2.0cm}

        Waterloo, Ontario, Canada, 2017 \\

        \vspace*{1.0cm}

        \copyright\ Peter Blouw 2017 \\
        \end{center}
\end{titlepage}

% The rest of the front pages should contain no headers and be numbered using Roman numerals starting with `ii'
\pagestyle{plain}
\setcounter{page}{2}

\cleardoublepage % Ends the current page and causes all figures and tables that have so far appeared in the input to be printed.
% In a two-sided printing style, it also makes the next page a right-hand (odd-numbered) page, producing a blank page if necessary.
 


% D E C L A R A T I O N   P A G E
% -------------------------------
  % The following is a sample Delaration Page as provided by the GSO
  % December 13th, 2006.  It is designed for an electronic thesis.
  \noindent
I hereby declare that I am the sole author of this thesis. This is a true copy of the thesis, including any required final revisions, as accepted by my examiners.

  \bigskip
  
  \noindent
I understand that my thesis may be made electronically available to the public.

\cleardoublepage

% A B S T R A C T
% ---------------

\begin{center}\textbf{Abstract}\end{center}

The most general goal of semantic theory is to explain facts about language use. In keeping with this goal, I introduce a framework for thinking about linguistic expressions in terms of (a) the inferences they license, (b) the behavioral predictions that their uses thereby sustain, and (c) the affordances that they provide to language users in virtue of these inferential and predictive involvements. Within this framework, linguistic expressions acquire meanings by regulating social practices that involve ``intentional interpretation,'' wherein people explain and predict one another's behavior through linguistically specified mental state attributions. I then propose an inferential role semantics to help explain \textit{how} linguistic expressions regulate intentional interpretation.

To describe this semantics, I take advantage of recently developed techniques in the field of natural language processing. I introduce a model that assigns inferential roles to arbitrary linguistic expressions by learning from examples of how sentences are distributed as premises and conclusions in a space of possible inferences. I then empirically evaluate the model's ability to generate accurate entailments for novel sentences not used as training examples. I argue that this model takes a small but important step towards codifying the meanings of the expressions it manipulates.

Next, I examine the theoretical implications of this work with respect to debates about the compositionality of language, the relationship between language and cognition, and the relationship between language and the world. With respect to compositionality, I argue that the debate is really about generalization in language use, and that the required sort of generalization can be achieved by ``interpolating'' between familiar examples of correct inferential transitions. With respect to the relationship between thought and language, I argue that it is a mistake to try to derive a theory of natural language semantics from a prior theory of mental representation because theories of mental representation invoke the sort of intentional interpretation at play in language use from the get-go. With respect to the relationship between language and the world, I argue that questions about truth conditions and reference relations are best thought of as questions about the norms governing language use. These norms, in turn, are best characterized in primarily inferential terms. 

I conclude with an all-things-considered evaluation of my theory that demonstrates how it overcomes a number of challenges associated with semantic theories that take inference, rather than reference, as their starting point. 


\cleardoublepage

% A C K N O W L E D G E M E N T S
% -------------------------------

\begin{center}\textbf{Acknowledgements}\end{center}

TBD - Thanks to all who helped.
\cleardoublepage

% T A B L E   O F   C O N T E N T S
% ---------------------------------
\renewcommand\contentsname{Table of Contents}
\tableofcontents
\cleardoublepage
\phantomsection    % allows hyperref to link to the correct page

% L I S T   O F   T A B L E S
% ---------------------------
\addcontentsline{toc}{chapter}{List of Tables}
\listoftables
\cleardoublepage
\phantomsection		% allows hyperref to link to the correct page

% L I S T   O F   F I G U R E S
% -----------------------------
\addcontentsline{toc}{chapter}{List of Figures}
\listoffigures
\cleardoublepage
\phantomsection		% allows hyperref to link to the correct page

% Change page numbering back to Arabic numerals
\pagenumbering{arabic}

